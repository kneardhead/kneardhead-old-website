\documentclass[12pt,a4paper,darkblue]{memoir}
\usepackage[article]{ahsan}
\usepackage{transparent}
\usepackage{pdfpages}


% figure support



\title{What is the Spherical Cow?}
\author{Ahmed Saad Sabit}
\date{\today}
\begin{document}
   
\maketitle

    \begin{small}
        This spherical cow is some sort of weird humour that physicists make. So physicists tend to simplify the models they are working with so that working on that becomes easy.
    \end{small}

    \section{How I want the doc to be, pre plan}
    
    \subsection{Aims of the Article}
    \begin{itemize}
        \item Make crystal clear what is Spherical Cow 
        \item Use this notion to make clear how a physicist would make calculations by simplyfing it, invoke a clue that physicists make things very simple at first.
        \item Give them an idea what working through a problem looks like and what are the real life challanges. 
        \item It will hopefully introduce a scientific meme. 

    \end{itemize} 

    \subsection{Structure of Document}
    \begin{enumerate}
        \item ``The notion of spherical cow comes from a Physicists inability to take account of all sorts of parameters while doing a calculation. Let me explain it" 
        \item joke.
        \item sacread. 
    \end{enumerate}

I am not sure how this will flow. 


\newpage 

\section{The Article}
Spherical cows are important to a Phycisist. So important that it even has it's own wikipedia page and has been mentioned by Stephen and Katherine Blundell in their Thermal Physics book.

Let me explain what's going on.

In reality it is just a humorous metaphor for simplifications that a Physicist makes when calculating complex real life phenomenon. 

At first I shall put the joke and then try to build the context.

``Milk production at a dairy farm was low. So, the farmer writes a letter to the local university, asking for help from Experts. 

For him a `multidisciplinary' team of Professors was assembled, the leader of the team was a Theoretical Physicist. 

The team started working, two weeks of intensive on-site investigation took place. The scholars then returned to the university, notebooks were full of data. And after the data were thoroughly analysed, it was sent to the Team Leader Physicist. 

Shortly the physicist returned to the farm, saying to the farmer, ``I have a fantastic solution, but it only works in the case of spherical cows in vacuum".



But wait! Physicists aren't that idiot and before you get to Facebook and demand to cut off the Physics Reasearch Grant by Government, let me now bring up the context. 

See, the daily life of a Theoretical Physicist is to solve problems that will make the lives of people easier. The job of finding out the Physics of Transistors led intel (trademark) company to make millions of dollars by selling us processor chips (you have better idea on this then me if you are building a Gaming PC). 


What it is like to solve a physics problem? It is not like doing an exercise from Shahjahan Tapan Sir's Physics textbooks. In real life the problems are not well defined, there is no rigid formulation that can help to solve it. The only tools leftover are idea, mathematical expertise and evidence. And sometimes the reality is confusing. Erwin Schrodinger came up with his Wavefunction equation about 20 years later of the Publication of Theory of Relativity. The wavefunction is supposed to describe small particles position and momentum, but problem was the equation didn't hold if the particles moved at high speed. 

In high speeds, you know, these time dilation and other facts cannot be ignored, but Schrodingers equation did not take them in account, this was a problem, small particles are often moving at super high speeds and if the Schrodingers equation, even after knowing Relativity, fails to fit with relativity, it will be a total nonsense.

Schrodinger went nuts to solve this blunder, but even though he tried very hard to fit things, it just didn't work. The corrections he tried had called infinities and other problems. But it is sure, that Schrodinger's Equation does describe correctly the properties of slow particles. 

In while doing such things like Schrodinger, Physicists cannot but rely on certain types of estimates and approximations, otherwise the mathematical framework becomes nearly impossible to work with. 

Let us just solve a problem and try to make sense. 

If you don't snooze like me in Physics class, then you know that a ball thrown at the sky at an angle takes a Parabolic shaped trajectory (trajectory is the shape of the path taken by a moving object) . Almost all Physics text book has the proof of this.

But this is false for real life cases, at least because it is an approximation. While deriving this result of parabolic shape of projectiles, we assume gravitational force caused by Earth is always straight pointing towards the flat ground. We also assume no other force than gravity exist.

Perhaps, these are not completely correct. First of all, Earth is round, and force points towards the center of the Earth's sphere shape. So, if you take account of Earth's curvature, then gravity points ``radially inward." So if the ball moves, then the direction of gravitational force will change. If you don't find this easy to swallow then just get a paper and draw a round Earth, then imagine a ball moving above the ground. This may clarify a bit. And other forces except gravity exist, it is the drag caused by air or centrifugal force because of Earth's motion. 

Drag caused by air is often very difficult to calculate, because it may depend on the temperature, angle what sun makes with ground, meterologic factors, wind and other geologic factors. Not just that, drag will depend on what outer texture of the ball is, whether it is rough or smooth, whether the ball is perfectly rigid or not. Jobs not done yet, there are also factors like Turbulent air flow around the ball, there is presence of ``added mass" air around ball, the portion of air that moves with the ball as it moves. 

Should I increase the problem a little more? The Earth is not a perfect sphere, its shape it's bumpy, called Geoid. And gravitational field of geoid is not same as sphere. Again, there is Astronomical intervening. Particles and other forms of radition from Cosmic rays may change the direction of ball slightly, there is this tidal forces by Sun and Moon (and other planets also), there is this centrifugal force that arises from the rotation around the Milky Way galaxy, there is also Coriolis force as the Earth rotates it's diurnal motion.

The ball's Electric and Magnetic Polarizability, Permittivity, Permeability and other Electromagnetic contents may interact with the Earth's Magnetic field and change the trajectory, sometimes very significantly. If it does so, then there may be changes in temperature of the ball, this may again interact with drag forces. 

Solving a problem of projectile isn't as easy.

This can even throw you in Philosophical crisis, you may even ask, ``Can't we just make the most accurate calculation ever?" I don't know what answer might be. But you don't need to get depressed of human's inability to calculate, a normal computer now can run any program that can take care of most of the factors and get us better than required numerical answers. And if a physicist is clever enough, he can just twick problems so that they become easier, like, when we want to consider Earth's curvature, then the projectile will be similar as planets, and keplar's law can be used for certain level of accuracy, we can dive deeper into it but let's just stick with spherical cows for now. 

% in middle of joke\ldots% in middle of joke...% in middle of joke...% in middle of joke.\ldots

We have dived too deep into Philosophical realms, this is a nice time to put another version of the same joke which is as popular,

``Once there was a farmer who was not happy with his milk farm and wanted to increase the milk production. He invited 3 people to check out what was going on!

The first one was a Psychologist – who observed the farm and told the farmer to paint the walls green, so that the cows will be happy and produce more milk.

The farmer thought: Huh, if life were that easy!

So he invited another person – the Engineer – who observed the farm and said, “The milking machine is not very effective. So I will design a new one for you.”

The farmer thought: Can I get a better perspective?

Well, now he invited a Physicist – who looked around the place and drew a spherical cow on the board saying, “Let me consider a spherical cow in a vacuum, emanating milk uniformly in all directions!”

This time the farmer and the cows were totally confused!" 

So, this should be accepted that a problem is usually made simpler with some level of approximations and simplifications. Perhaps you might be thinking, oh no, then we can't get perfect results! 

But this doesn't verify that these calculations are wrong, sometimes they amaze with their fantastic correspondence to real results, even though the approximations are very naive. How?

Few weeks ago a paper by Andrew Lucas became trending amongst the Physics Nerds online. The title is bit long though. I won't mention it here for now (check reference for it). There, using very simple estimates a bit too much accurate results were obtained. 

This will make clear, let us think of the densest matter of the world, what can be it's density? 

Let us make a very rough estimate. Suppose that the matter were made of Hydrogen atoms. Mass of Hydrogen atom can be found from the back of your chemistry text book (appendice or data sheet). And it's Bohr radius can be calculated from equations. Now, assume the Hydrogen atoms are packed tightly. Very tightly, each atoms are packed, as if a lot of cube shaped boxes are stacked and put together to build a larger cubical box. 

So each small cube contains an atom, and let's just roughly assume each side of cube has length of Bohr Radius for Hydrogen atom. You can draw it on a paper. 

Now if you calculate the density of such a small cube, it is, 16000 kg per meter cubed. 

Remember we packed the atoms very tightly, and thus it is densest in our estimates. But this surprisingly is close to the real densest matter in the World, it is Osmium, which has a density of 22000 kg per meter cubed, not very far from 16000!

Spherical cows are ever present in doing Physics, I don't know about other sciences, they may have their own version as well. Approximations make the theory doable, but usually, these approximations are better than what we need. 

I like to think it as a sort of a "blessing" or just a "present" for human beings. 

If nature had the structure that made approximations not as good as we do now, then Human probably be nowhere relative to what it is now.
































\end{document}
