

\documentclass[10pt, a4paper]{memoir}
\usepackage[fancythm, lightgreen,nosecthm]{ahsan}


% figure support

\usepackage{import}
\usepackage{xifthen}
\usepackage{pdfpages}
\newcommand{\incfig}[1]{%
    \def\svgwidth{\columnwidth}
    \import{./figures/}{#1.pdf_tex}
}



\title{The General Matrix problem of Oscillation}
\author{Ahmed Saad Sabit}
\date{\today} 

\usepackage{mathdots}
\counterwithin{figure}{section}

\begin{document}
\maketitle


\prob{  }{  }{  }{ There are two masses $m_1$ and $m_2$ that are positioned between two walls in a coupled system. Find the modes of oscillations \emph{using Matrices}. }

\solu{ 
        For a displacement, $x_1$ nad $x_2$, 
        For the first mass,
        \[ 
            -k x_1 + k(x_2 - x_1) = - k (2x_1 - x_2)
\] For the Second mass,

\[ 
    - k \left( x_2 - x_1 \right) - k x_2 = - k (2x_2 - x_1)         
\]
The equation of motion,
\[ 
    \ddot{x_1} = - \frac{k}{m} \left( 2x_1 - x_2 \right) 
\]
\[ 
    \ddot{x_2} = -\frac{k}{m}\left( 2x_2 - x_1 \right) 
\]

Assume that the solution of these are,
\[ 
\begin{pmatrix} x_1 \\ x_2 \end{pmatrix} = 
e^{i \alpha t} 
\begin{pmatrix} A \\ B \end{pmatrix} 
\]
Thus, 
\[ 
    \ddot{x_1} = - \alpha^2 A e ^{i\alpha t} = - \alpha^2 x_1 = - \omega^2 \left( 2A - B \right) e^{i \alpha t}
\]
\[ 
    \ddot{x_2} = - \alpha^2 B e ^{i \alpha t} = -\alpha^2 x_2 = - \omega^2 \left( 2B - A \right) e ^{i \alpha t}
\]

From there, we can write,
\[ 
- \alpha^2 A + 2 \omega^2 A - \omega^2 B = 0
\]

\[ 
-\alpha^2 B + 2 \omega^2 B - \omega^2 A = 0
\]

In matrix form,
\[ 
    \begin{pmatrix} -\alpha^2 + 2\omega^2 & -\omega^2 \\
        -\omega^2 & - \alpha^2 + 2 \omega^2
    \end{pmatrix} 
    \begin{pmatrix} A \\ B \end{pmatrix} =0
\]
This requires the $\det m$ to be zero,
\[ 
    \left( - \alpha^2 + 2\omega^2 \right) ^2 - \left( \omega^2 \right) ^2
\]
\[ 
= \alpha ^{4} + 4 \omega^{4} - \omega^{4} - 2 \alpha^2 \cdot  2 \omega^2 
\]
\[ 
    \alpha ^{4} - 4 \alpha^2 \omega^2 + 3 \omega ^{4} 
\]

This is equal to zero,
\[ 
\alpha ^{4} - 4 \alpha ^2 \omega^2 + 3 \omega ^{4} = 0
\]
This solves, 
\[ 
\alpha^2 = \frac{4 \omega^2 \pm \sqrt{16 \omega^4 - 12 \omega^{4}} }{2}
\]
The solutions are,
\[ 
\alpha = \pm \omega
\]

   }


\end{document}
