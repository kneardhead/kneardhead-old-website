\documentclass[12pt,a4paper]{memoir}
\usepackage[book, nosecthm]{ahsan}
\usepackage[]{verbatim}

\author{Ahmed Saad Sabit}
\date{03 July, 2020}
\title{Polarization and Electric Field inside material from that.}
\begin{document}
\graphicspath{{\Pics}}
 \maketitle
 \section{A Dipole's General Behaviour}
So, in a material, all the molecules are some dipoles oriented randomly. A dipole is a positive negative charge pair, when they are separated by $d$  and the charge of each (positive magnitude) is $q$, then dipole moment is $\vec{p} = q \vec{d}$, $d$ is negative to positive charge vector. Well that's how it comes. Remember, the torque generated by an external $\vec{E}$ field is,
\begin{equation}
\vec{\tau} = \vec{p} \times \vec{E}
\end{equation}
At any $r$ from the dipole, $\theta$ angle,
\begin{equation}
V = k\frac{p \cos \theta }{r^2}
\end{equation}
And in vector form,
\begin{equation}
V = k \frac{\vec{p} \cdot \hat{r}}{r^2}
\end{equation}
Well, electric field is readily found for a dipole using a simple gradient.
\begin{equation}
\vec{E} = - \nabla V = k \frac{3 ( \vec{p } \cdot \hat{r}  ) \hat{r}  - \vec{p} }{r^3}
\end{equation}
This is all the basic. \\
\section{Dipoles in Materials: Polarization}
Well, we can have materials that are \emph{Polarized}. In fact, magnets if thought ``electricish" are well suited example. So goes with an Electrically Polarized Material, an ``Electricity Bar Magnet" roughly speaking. 

We are interested in the internal features of it. Let there be $n$ dipoles inside the material per volume. It is a mere number. So, dipoles in 1 $m^3$ is $n$. 

Then, if each small dipole in the material is $\vec{p}$ moment, then, net dipole moment of a $m^3$ material is going to be,
\[ \vec{P} = n \vec{p} \]
To be simple, assuming directions known, we can say,
\begin{equation}
P = nq d
\end{equation}
Here, $P$ is known as the \emph{Polarization} of a material, which is the, \textbf{Dipole Moment per Volume.}
\begin{figure}[ht!]
    \centering
    \includegraphics[width=0.8\textwidth]{Pics/polarizer.png}
    \caption{}
    \label{fig:}
\end{figure}
For a material, if we are able to find a net dipole moment overall, the dipoles are oriented somewhat in the manner that produces a dipole moment. This is justified in this manner, the internal dipole moments all cancel out each other. But, the number of small dipoles that are at the boundary are somewhat free to make a field happen. And makes sense, the dipole moment can be made by a cube if two opposite surface has opposite surface charge $\sigma$ per area, not volume! 

We are interested of the Charges \textbf{inside the material}, not outside. Then, for a small cube with coordinate $(x_1,x_2,x_3)$, the sides length $dx_1,dx_2,dx_3$.

The dipole moments at the surface at $x_1 + dx_1$ make positive charges lag out and other side at $x_1$ makes negative charges lag out.\textit{ So inside remains negative charge for first case and negative charge for the second.}

Total charge if we consider these sides, that makes a dipole relative to \textit{inside},
\[dq(x_1) =  (nq) d \, dx_2 \,dx_3 = P(x_1) dx_2 \,dx_3\]
And for the other side, in similar way,
\[dq(x_1 + dx_1) = - (nq) d \, dx_2 \,dx_3 =- P(x_1 + dx_1) dx_2 \,dx_3\]
So, total charge for case of axis $x_1$,
\[ dq(x_1)+dq(x_1 + dx_1)= P(x_1) \, dx_2 \,dx_3 -P(x_1 + dx_1) dx_2 \,dx_3 \]
Hopefully this reduces nicely, knowing that $ P(x_1 + dx_1) - P(x_1) / dx_1 = \partial P/ \partial x $
\begin{equation}
dq_{x1} =- \frac{ \partial P}{\partial x_1} \, dx_1 \, dx_2\, dx_3
\end{equation}
For all surface if we generalize for now,
\begin{equation}
\sum_{surfaces} dq_{p} = \left(  \frac{ \partial P}{\partial x_1} +
 \frac{ \partial P}{\partial x_2} +
  \frac{ \partial P}{\partial x_3}   \right) \, dx_1 \, dx_2\, dx_3
\end{equation}
Which is known as,
\begin{equation}
\rho _p = - \nabla \cdot \vec{ P }
\end{equation}
In one manner, in the flow of polarization, the influx or outflux is equal to the density of charges that are part of the internal polarization.

Now if we think some materials are permanently polarized and think some become polarized in presence of an electric field. With the above analysis, we don't care \textbf{how the polarization came}, we found \textbf{what shall happen} if there's polarization.

\nt{\begin{small}
For more understanding, let me (I am confused still) think that we have made a block, a lego block like thing from many small Rubik's Cube pieces like things. Inside goes many dipoles. Now, each of the small blocks have a \emph{Polarization per unit volume}, what if for all the small blocks it is perfectly oriented? The matter will become the best dipole in the world. If there is misorientation, then the $P_{x_1}, P_{x_2}, P_{x_3}$ will radically change, this change can only be bought be some (what?). Well, I would like to view the $P$ as a polarization density, but a density being a $vector$ is hard for me to understand. Well, if dipoles arrange, then there will be net electric field. 

I again think that this weird charge density shall give rise to an electric field that we have to take account for. And this is the field that will stay inside the material apart of the isolated charge that we are used to. 
\end{small}}
Okay, so total charge density of a material is, 
\begin{equation}
\rho + \rho _p = \rho - \nabla \cdot \vec{P}
\end{equation}
I can come to that later, let us define a much important thing beforehand, 
\begin{equation}
q_p = \int_{material} \rho_p \, dV = -\int \nabla \cdot \vec{P} \, dV = -\oint_{S} \vec{P} \cdot \, d\vec{S}
\end{equation}
In similar manner,
\begin{equation}
q =\epsilon_0 \int_{material} \rho \, dV = \epsilon_0\int \nabla \cdot \vec{E} \, dV = \epsilon_0 \oint_{S} \vec{E} \cdot \, d\vec{S}
\end{equation}
Polarized Charges are internal part to the material. Then the free charge is given by the difference of the overall charge $q$ and polarized charge (personal electrons of material) $q_p$. So, we shall define free charges by,
\[ q_{free} = q_{net} - q_{polar} \]
So, 
\[q_{free} =   \oint_{S} \epsilon_0 \vec{E} \cdot \, d\vec{S} -\oint_{S} -\vec{P} \cdot \, d\vec{S} \]
Nicely writing,
\begin{equation}
  \oint_{S} \epsilon_0 \vec{E} \cdot \, d\vec{S} +\oint_{S} \vec{P} \cdot \, d\vec{S} = q_{free}  
\end{equation}
Remember, \textbf{negative sign in front of Polar formula!} \\
To make the above formula look more beautiful, we shall make it symmetric. Let us assign a vector $\vec{D}$ to integrate over the surface to get the free surface. Which is,
\begin{equation}
  \oint_{S} \epsilon_0 \vec{E} \cdot \, d\vec{S} +\oint_{S} \vec{P} \cdot \, d\vec{S} = \oint_{S} \vec{D} \cdot d\vec{S} 
\end{equation}
And we have (start drumming...),
\begin{equation}
\epsilon_0 \vec{E} + \vec{P} = \vec{D}
\end{equation}
\nt{I will think in this manner. Then density whatever, for the reason,
\begin{equation}
\vec{E} + \frac{\vec{P}}{\epsilon_0} = \frac{\vec{D}}{\epsilon_0} 
\end{equation}
So, we have some fields, in another sense, total field minus polarization field equals to free field. Or, I want to say,
\[ E = \frac{D}{\epsilon_0} - \frac{P}{\epsilon_0} \]
The free charge field subtracted by the polarization field yields the field of the total system, inside the material. Fine. But how fine. Grrrrrrrrrrrr, Matha kharap koirao sarbe na eto sohoze. I now even want to assign potentials to them and see what happens. 
}The $\vec{D}$ is the Displacement Vector. For cases, we can bear the fact that, whenever material is present, we given with some resisting $P/ \epsilon_0$ field (linear arki), then the rest of the field that is remaining is found from the free charges, don't care where they are, either in the material or space. Well, in free space, there is nothing to polarate and for that,
\begin{equation}
\vec{D} = \epsilon_0 \vec{E}
\end{equation}
$D$ is about being independent. $E$ is the net overall.
\section{Polarizing Source}
Put something in an $E$ field, what we get is that the material is polarized as the molecules feel the torque we describe in the beginning. $E$ field enters in material as it is, but the polarization is the one who changes it. We will say that the $E$ given at place is $E_{local}$.

So, the dipole moment $\vec{p}$ arising from a local field $\vec{E}_{local}$ is given by,
\begin{equation}
\vec{p} = \alpha \vec{E}_{local}
\end{equation}
We make the local field, we give it and the material changes itself from being torqued by it. Now, inside the material, there should be an electric field too. \textbf{If the shape is a sphere}, we have this that, derived,
\begin{equation}
< \vec{E}_{in} > = -\frac{\vec{P}}{3 \epsilon_0}
\end{equation}
We cannot but take average of field inside the dipole system as the charges change field drastically, all we have is the average effect in the volume inside the dipole. This average is the contribution of the inside charges only. Now, total field inside the system,
\begin{equation}
\vec{E_{in}} = <\vec{E}> + \vec{E}_{local} = \vec{E}_{local}  - \frac{\vec{P}}{\epsilon_0}
\end{equation}
Now, we can redefine a few things,
\[ P = np = n \alpha E_{local} = n \alpha 
\left( E + \frac{P}{3 \epsilon_0} \right) \]
That gives that, if we solve for $P$, and I have ignored vector
notation for now, but do assume vector where needed, a 
\verb| \vec{P} | is painful to write. 
\begin{equation}
P = 
\frac{n \alpha}{1- n \alpha / 3 \epsilon_0} E 
= 
\chi \epsilon_0 E
\end{equation}
Redoing the math,
\begin{equation}
\vec{D} = \epsilon_0 \vec{E} + \vec{P} 
=
\epsilon_0 (1 + \chi) \vec{E}
= 
\epsilon_0 \epsilon_r \vec{E} =
\epsilon \vec{E}
\end{equation}
So, permittivity of the material is known, that is the 
relative times normal permittivity of vacuum we are used to. 
\begin{equation}
\vec{D} = \epsilon \vec{E}
\end{equation}
   
\theo{}{Equation of Polarization}{We can finally put the whole equation and the definitions of the terms in it below.
\begin{itemize}
\item \textbf{E is for Total Final Electric Field} that we shall get after the whole system has came to the stable form. It contains all the perturbations added perfectly. 
\item \textbf{P is the Polarization per Volume } that gives the \textbf{inner charges} of the material and this $P/ \epsilon _0$ can disturb the field what we give inside the material. $P=0$ for vacuum.
\item \textbf{D is for the case when we assume there was no material}, that is the full system was just independent and free, we are in space.
 \end{itemize}}

\section{So How do we use it?}

\prob{}{}{}{Imagine that we have a sphere that is made out of some material, of course not a conductor. Now, if we have some $q$ charge in the center of the sphere, then, what can be the surface charge induced?}
This is not too simple for me to think to think simply as it is not seen too simply by me. 
\\
\solu{The $\vec{D}$ is the ambassador of things that are analyzed in the free space. For this reason, we can tell when it was a free space, 
\[ \oint \vec{D} \cdot d\vec{S} = q \]
That shall reduce down to,
\begin{equation}
D = \frac{q}{4 \pi r^2}
\end{equation}
Now, we know, that,
\[ D = \epsilon E \]
So, 
\[ E = \frac{D}{\epsilon} \]
Of course this $\epsilon $ belongs to the material. Now, the field inside the material will be,
\[ E = \frac{q}{4 \pi \epsilon r^2} \]
The limit $ r < R$ radius of sphere.
So, now we can write, 
\[\vec{P} =  \vec{D} - \epsilon_0 \vec{E} = \vec{D} \left(
1- \epsilon_0 / \epsilon
\right)
=
\frac{(\epsilon - \epsilon_0 ) q}{4 \pi \epsilon r^2}
 \]
So, 
\[ P =
\frac{(\epsilon - \epsilon_0 ) q}{4 \pi \epsilon r^2}
 \]
Near the charge will sit the opposite polarity things, and outside, as $\sigma = P$, so the magnitude of the induced surface charge is going to be simply,
\[ \sigma = 
\frac{(\epsilon - \epsilon_0 ) q}{4 \pi \epsilon r^2}
 \]
The volume polarization charge tends to be zero in this case. 
\[ \nabla \cdot \vec{P} = \frac{1}{r^2} \frac{\partial }{\partial r} (r^2 P_r) = \frac{1}{r^2} 
\frac{\partial }{\partial r} \frac{(\epsilon - \epsilon_0 ) q}{4 \pi \epsilon}\]
or,
\[ \rho _p = -  \nabla \cdot \vec{P} = 0 \]
There is no net charge making volume in the system. But this might not be the case the $\epsilon$ was not a constant but $\epsilon(r)$.}
\textbf{Remark:} $\epsilon$ looks small but it needs to write \verb|$\epsilon$| for this. Highly time consuming, as I don't yet use the enter key. Well, the appropriate sign was $\varepsilon$, that is \verb|\varepsilon|. Well, so far so good, $\varepsilon$ $\eta o \tau$ $\alpha$ $\Gamma$rea$\tau$ thing.

































\end{document}
